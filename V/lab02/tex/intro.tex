\chapter*{Введение}
\addcontentsline{toc}{chapter}{Введение}
RISC-V является открытым современным набором команд, который может использоваться для построения как микроконтроллеров, так и высокопроизводительных микропроцессоров. В связи с такой широкой областью применения в систему команд введена вариативность. Таким образом, термин RISC-V фактически является названием для семейства различных систем команд, которые строятся вокруг базового набора команд, путем внесения в него различных расширений.

\textbf{Цель работы}: ознакомление с принципами функционирования, построения и особенностями архитектуры суперскалярных конвейерных микропроцессоров, а также знакомство с принципами проектирования и верификации сложных цифровых устройств с использованием языка описания аппаратуры SystemVerilog и ПЛИС.
Для достижения данной цели необходимо выполнить следующие задачи:
\begin{itemize}
	\item ознакомиться с набором команд RV32I;
	\item ознакомиться с основными принципами работы ядра Taiga: изучить операции, выполняемые на каждой стадии обработки команд;
	\item на основе полученных знаний проанализировать ход выполнения программы и оптимизировать ее; 
\end{itemize}

В настоящей лабораторной работе используется синтезируемое описание микропроцессорного ядра Taiga, реализующего систему команд RV32I семейства RISC-V. Данное описание выполнено на языке описания аппаратуры SystemVerilog.

Все задания выполняются в соответствии с вариантом №4.