\chapter*{Задание 1}

Практикум 1 выполнялся по варианту 4.

\textbf{Цифровой интерполятор ЧПЗ.} Сформировать в хост-подсистеме и передать в SPE 256 значений x и функции f(x)=sin(x), имеющие тип double (где x - ключ, f(x) - значение). Для представления чисел double в целочисленном диапазоне использовать функции double ull2double(uint64\_t) и uint64\_t double2ull(double), входящие в библиотеку sw\_kernel-lib. Для случайного значения, сформированного в хост-подсистеме выполнить поиск ближайшего большего, и передать его в хост-подсистему. Выполнить тестирование работы SPE, сравнив результат с ожидаемым.

Для выполнения данного задания были изменены файлы с кодом примера

В листинге \ref{listing_create} представлено создание и передача данных из буфера.

\begin{center}
	\begin{lstlisting}[label=listing_create,caption=Создание и передача данных из буфера]
	__foreach_core(group, core)
{
	for (int i=0;i<BURST;i++) {
		host2gpc_buffer[group][core][2*i] = f_rand(0, 10);
		host2gpc_buffer[group][core][2*i+1] = rand_number;//i;
		printf("key: %lf, value: %lf\n", host2gpc_buffer[group][core][2*i], host2gpc_buffer[group][core][2*i+1]);
	}
}

__foreach_core(group, core) {
	lnh_inst.gpc[group][core]->start_async(__event__(insert_burst));
}

__foreach_core(group, core) {
	lnh_inst.gpc[group][core]->buf_write(BURST*2*sizeof(double),(char*)host2gpc_buffer[group][core]);
}

__foreach_core(group, core) {
	lnh_inst.gpc[group][core]->mq_send(IP);
}
	\end{lstlisting}
\end{center}

В строках 2--8 создаётся буфер с данными, который далее передаётся в глобальную память и загружается в Тераграф с помощью функции \textit{insert\_burst} (листинг \ref{listing_insert}).

\begin{center}
	\begin{lstlisting}[label=listing_insert,caption=Создание и заполнение графа]
void insert_burst() {
	
	lnh_del_str_sync(TEST_STRUCTURE);
	unsigned int count = mq_receive();
	unsigned int size_in_bytes = 2*count*sizeof(uint64_t);
	uint64_t *buffer = (uint64_t*)malloc(size_in_bytes);
	buf_read(size_in_bytes, (char*)buffer);
	for (int i=0; i<count; i++) {
		lnh_ins_sync(TEST_STRUCTURE,buffer[2*i],buffer[2*i+1]);
	}
	lnh_sync();
	free(buffer);
}	\end{lstlisting}
\end{center}

При этом используется функция \textit{mq\_send}, выполняющаяся на ядре (листинг \ref{listing_send}).

\begin{center}
	\begin{lstlisting}[label=listing_send,caption=Создание и заполнение графа]
		void get_interface()
		{
			lnh_sync();
			unsigned int key = mq_receive();
			lnh_search(TEST_STRUCTURE, key);
			mq_send(lnh_core.result.value);
	}	\end{lstlisting}
\end{center}

\clearpage
Исходя из переданных данных, производится интерполяция (листинг \ref{listing_test}).

\begin{center}
	\begin{lstlisting}[label=listing_test,caption=Интерполяция переданных  величин]
void search_burst() {
	lnh_sync();
	unsigned int count = lnh_get_num(TEST_STRUCTURE);
	unsigned int size_in_bytes = 2*count*sizeof(double);
	double delta = 0;
	double closest = DBL_MAX;
	double *buffer = (double*)malloc(2 * sizeof(double));
	double *old_buffer = (double*)malloc(size_in_bytes);
	double min_delta = DBL_MAX;
	lnh_get_first(TEST_STRUCTURE);
	for (int i=0; i<count; i++) {
		old_buffer[2*i] = ull2double(lnh_core.result.key);
		old_buffer[2*i+1] = ull2double(lnh_core.result.value);
		lnh_next(TEST_STRUCTURE,lnh_core.result.key);
	}
	double number_to_interpolate = old_buffer[1];
	for (int i=0; i<count; i++) {
		double cur = old_buffer[2*i];
		if (old_buffer[1] >= old_buffer[2*i])
		delta = number_to_interpolate - cur;
		else
		delta = cur - number_to_interpolate;
		if (delta < min_delta) {
			min_delta = delta;
			closest = cur;
		}
	}
	buffer[0] = old_buffer[1];
	buffer[1] = closest;
	buf_write(size_in_bytes, (char*)buffer);
	mq_send(count);
	free(buffer);
	free(old_buffer);
}

\end{lstlisting}
\end{center}
